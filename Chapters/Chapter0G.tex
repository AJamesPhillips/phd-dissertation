% Appendix G

\chapter{Health 0.0}
\label{appendix:health}

\textbf{Readouts and reports from individual workshops:} \\

% Workshop 1
\textbf{Workshop 1: October 7th 2016.}
 
A group of Life sciences professionals held a one-day meeting with the MIT Media Lab to discuss the potential to develop a new paradigm for drug development. The meeting examined a new drug development approach, which is prompted by a changing healthcare environment resulting from the adoption of new technologies.

It was proposed, that the following factors need to be considered:
\begin{itemize}
\item Pharmaceutical Industry, Providers and Payers would have to develop a robust, consistent and dynamic data sharing environment
\item Clarity would be needed on who owns data and how broad access is achieved to accelerate research
\item New incentive models would need to be developed across multiple stakeholders in the clinical trial ecosystem to ensure strong participation and data sharing
\item A new paradigm on potential liability will need to be introduced requiring new models of indemnification (e.g. joint liabilities)
\item Regulators would have to be open to the new model and build the associated needed capabilities and resources
\end{itemize}

Overall the group, felt that the development of a new model was worth further exploration and that other interested parties should be engaged. It should be noted that if the new model were successful, it would have potential benefits for the patient and key stakeholders. The team should also assess what other efforts are being pursued on this topic by other entities e.g. NewDigs, EMEA, Transcelerate etc.

Full report from the workshop: \href{https://drive.google.com/file/d/1kFMbGtNa5IxEZz02Qn8uw-oOeWQ1qIiq/view}{Workshop 1 full report} \\

% Workshop 2
\textbf{Workshop 2: March 17th 2017}

A group of Life sciences professionals from member companies held a 1-day meeting with MIT Media Lab to discuss approaches to apply and develop new Machine Learning (ML) and Deep Neural Networks (DNNs) for classification of clinical and pharmaceutical research data. Pratik Shah from the Media Lab presented a two-hour overview of the current state of Artificial Intelligence (AI) in computer science and related fields, followed by five case studies of emergent AI for classification of clinical and biological data.  Francis Kendall from Roche and representatives from Biogen, Microsoft, Medimmune and VSP then shared individual case studies, challenges and machine learning approaches being used in their respective companies.Three potential areas for engagement with the Media Lab and other member companies were discussed:
1) New algorithms for automated structuring of raw biological data for input into ML and DNN classifiers by bioinformatics and data science professionals;
2) Develop new models and emerging DNN architectures for classification of multimodal clinical and biological datasets
3) How do we build a horizontal pharma/bio data platform that all media lab member companies can contribute to and get value from? There is sustainable interest from the Media Lab pharma companies to continue similar workshops to discuss the intersection between AI and drug discovery, but for logistical reasons these will be combined with the regulatory drug discovery  workshops. \\
 
\textbf{Key recommendations and agreed actions}
\begin{enumerate}
\item a. Publish key challenges of the new approach to start a dialogue for the wider community to solve \\
b. Data – Explore the feasibility of an Open Data Model \\
c. Develop collaboration experimentation e.g. Sharing data. \\
d. Predictability - How Might We apply AI/Machine Learning to enhance the process?
\item Additional examples of published successful DNN/ML approaches to be shared with team
\item Pharmaceutical company representatives will explore opportunities to identify and agree on a data collaboration study
\item Explore and link to ongoing conversations and initiatives on the new paradigm for clinical development theme.
\end{enumerate}

Full report from the workshop: \href{https://drive.google.com/file/d/1l1dgWlCuBms-LWHdrP6-4qDdLJs83fkq/view?usp=sharing}{Workshop 2 full report} \\

% Workshop 4
\textbf{Workshop 4: October 10th 2017}

The morning was a forum with presentations various topics related to engendering a new digital paradigm for the future of health and drug development. Speakers from pharma, health, technology Media Lab Member companies and the FDA participated. The afternoon will include a workshop facilitated by The Boston Consulting Group where speakers and guests can brainstorm. Followed by an executive meeting of key leaders and speakers from various organizations to chart out next steps:
Objectives included:
\begin{itemize}
\item Developing a sustainable model to bridge the gap between AI and data science experts and the life sciences community
\item Addressing current, near-term AI, machine learning, and neural network capabilities as they pertain to drug development
\item Identifying new AI inventions and data structures that can solve key drug development challenges
\item Identifying collaborations around use cases for existing AI to solve high-impact drug development challenges
\item Discussing roadblocks limiting the full potential of AI in drug development
\end{itemize}

Final workshop agenda: \href{https://www.media.mit.edu/events/artificial-intelligence-in-clinical-development-to-improve-public-health/}{Workshop 3 agenda}

\textit{Key action items from this workshop (listed below) are been prepared in the form of a Perspective article to be submitted to Nature Reviews Drug Discovery} \\

% TODO
\textbf{Key outcomes and next steps:}

\begin{enumerate}
\item The first priority is data aggregation an availability
\begin{itemize}
\item e.g. Common hurdle: Significantly limited amount of high-quality, sufficiently large datasets (especially for AI training)
\item Potential approaches: Members can start by contributing legacy "abandoned project" or pre-competitive, masked datasets; unstructured and structured data should be collected and pre-processed; new data capture standards (format, access etc.) should be defined
\end{itemize}

\item AI can be used for aggregation AND (retrospective and prospective) analytics
\begin{itemize}
\item AI is not just for analysis, but also for data pre-processing/aggregation
\item Confidentiality or competition-sensitive concerns can be addressed using a decentralized aggregation and analysis approach
\item Incentivisation between the AI and Pharma community needs to find common ground to accelerate collaboration
\item Analytics should include retrospective analysis and efforts to capture new, novel and better suited data to enable prospective analysis
\item A focus is needed on pervasive, foundational enablers in order to truly enable use of AI
\item We should shift our conversations to include AI when we talk about Digital Health and RWD – it's all about data and solution ecosystems
e.g. Pervasive integration of Digital Health tools along the care journey ("at scale"), i.e. point of care to real world setting
\item Pharma and Payer members to identify Digital Health integration points along the care journey in which to embed Digital Health tools for the purpose of, but not limited, enhanced data collection to enable extensive feature rich datasets for AI-based analytics
\item Standardized analytical approaches are uncommon and transparency on approaches is limited
\item Multi-stakeholder alignment, awareness, education activities and ethics discussions are required to realize the adoption, integration and ideal leverage of AI in Clinical Development and healthcare in general
\item Conservatism and the need for education and alignment for regulators, payers, clinicians, and patients (vs. ``black box'' problem, acceptance of non-traditional measures)
\item Roles of ethics associated with AI in healthcare, especially in Pharma setting, needs to be more clearly explored and defined
\end{itemize}
\end{enumerate}