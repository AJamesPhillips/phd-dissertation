% Appendix A

\chapter{Professor of the Practice Research Statement}
\label{ch:researchstatement}


\emph{Following is the research statement that I presented to the Massachusetts Institute of Technology for my promotion case to Professor of the Practice in Media Arts and Science in March of 2016. The statement describes my vision for the Media Lab and my role. The case was successful.} \\

\textbf{Research Statement} \\

Science, engineering, design, and art can together be viewed as a circle where the output of one is the input of another. Design and science are opposite each other on this circle; that is, the output of one is not the input of the other, as is often the case with engineering and design, or science and engineering. I believe that by making a ``lens'' that fuses design and science, we can fundamentally advance both. 

When I joined the Media Lab in 2011 as its director, ``antidisciplinary'' was a word new to me. It was listed as a requirement in the faculty search description. Antidisciplinary, as opposed to ``interdisciplinary,'' is about working in spaces outside the traditional academic disciplines --- about a new way of working without the traditional tools, such as a focus on a particular scale and a specialized language, that typify the current scholarly research. By bringing design and science together, we can foster new, productive, and flexible antidisciplinary work. 

In many ways, the cybernetics movement of the 1940s and 1950s has served as a model for what the Media Lab does --- drawing on new technologies to create a movement cutting across disciplines. Cybernetics spawned many exciting new disciplines but, as a field, it fragmented, with the offspring of the intellectual leadership ending up in many of these newly established fields. My aim is to develop a movement that is agile, engaging, and antidisciplinary enough not only to survive but to thrive on its own even as new disciplines spin off.

We are looking for people who don't fit into any existing field of study. I often say that if you can do what you want to do in any other lab or department, you should go do it there; come to the Media Lab only if there is nowhere else for you to go. We are the new \textit{Salon des Refusés}.

Another analogy for the space I'm promoting: Think of a sheet of paper representing the whole of science. Its various fields are black circles on this paper. The white space between circles represents antidisciplinary space. Many people would like to play in it, but there is very little funding for them; moreover, it's often hard to get a tenured position without some sort of anchor in one of the disciplinary circles. 

The Internet and increasingly more powerful computational tools, accelerated the rate at which research can be conducted, shared and combined. This has generated a new opportunity but also increases the complexity, making it increasingly difficult to tackle many of the interesting problems through a traditional disciplinary approach. Unraveling the complexities of the human body is a perfect example. Brain research, for instance, involves a diversity of disciplines, among them computational optics, nanotechnology, data science, systems biology, and the microbiome. Therapeutics are just as diverse, encompassing pharmacology, electromagnetic interactions, and opto-genetics as well as nanotechnology. Traditional interdisciplinary research involved bringing researchers together across these disciplines, whereas increasingly, the key researchers are able to straddle multiple disciplines and translating and synthesizing in a way that is difficult or impossible as a conversation between disciplines. Many current efforts seem unable to move beyond a mosaic of so many disciplines that often we don't realize we're all looking at the same problem, so different are our methods, our instruments, and our language. 
While the space between and beyond the disciplines can be academically risky territory, it allows for promising unorthodox (and often cheaper) approaches that draw on hitherto insular regimes. The Internet's enabling of collaboration at nearly no cost, as well as the diminishing costs of computing, prototyping, and manufacturing, also contribute to the flexibility of antidisciplinary research. \\ 

\textbf{Addressing the World Through This New Lens} \\

Whereas science arguably moves toward this antidisciplinary convergence, design has become what Marvin Minsky, in The Emotion Machine, calls a ``suitcase word'': It means so many different things that it means effectively nothing. Design, as I use it here, refers to the iterative process of understanding the constraints of a system and creating something that will have an effect on it. Unlike engineering, design is not as much about solving problems as asking questions. The designer is the architect, the maker, the scientist who introduces a new point of view. It encompasses many important ideas and practices, and thinking about the future of science in the context of design promises to be a fruitful endeavor.

Design has progressed from the design of objects to the design of systems and on to the design of complex adaptive systems. This evolution is shifting the role of designers, who should be seen not as planners apart from, but as participants within, the systems they exist in. Today they work for companies or governments, developing products and systems focused primarily on ensuring that one or another aspect of our society works efficiently, with scant concern for systems beyond specific corporate or governmental needs. But we're moving into an era in which system boundaries are not as defined. These underrepresented systems—the microbial world, say, or the global climate, or the environment—present significant design challenges. They are self-adaptive complex systems, and our unintended and unexamined effects on them will most likely have negative consequences for us.

Media Lab professor Neri Oxman and architecture professor Meejin Yoon co-teach a popular class called ``Design Across Scales,'' in which they discuss design at scales ranging from the microbial to the astrophysical. While it is impossible to predict the outcome of complex self-adaptive systems, we can indeed hope to understand and take responsibility for our interventions within them. This is designing absent the ability to control-—more like giving birth to a child and influencing its development than designing a robot or a car.

An example of this kind of design is the work of Media Lab professor Kevin Esvelt, who describes himself as an evolutionary sculptor. He is working on ways of editing the genes of populations of organisms such as the rodent that carries Lyme disease and the mosquito that carries malaria, to make them resistant to those pathogens. His aim is to effect a change in the genome that will spread throughout the entire population of the organism. Thus its consequences will alter the whole ecosystem, including the biosphere, public health policy, and the ethical issues attendant on these sorts of interventions. What is novel here is consideration of the effects of a design on all of the systems that touch it.

When the cybernetics movement began, the focus of science and engineering was on such tasks as guiding a ballistic missile or controlling the temperature in an office --- problems squarely in the man-made domain and simple enough to yield to traditional siloed methods of scientific inquiry. For those of us working in the contemporary space of design and science, there are no obvious boundaries in the territory we are addressing. Formerly there was a clear separation between the artificial and the organic, the cultural and the natural. Today, the man-made and the natural are no longer separate --- they are one. 
Science and engineering now pursue problems in synthetic biology and artificial intelligence --- undertakings so complex they extend beyond the domains of existing disciplines. We are finding that in many ways we can now ``design'' nature. Artificial intelligence, for example --- a digital rather than a natural science --- is moving beyond a merely metaphorical relationship to the human brain. By picking up where cybernetics left off and redirecting the development of modern design to the future of science, we believe that a new kind of design and a new kind of science may emerge, and in fact is already emerging. \\

\textbf{Rethinking Traditional Academic Research} \\

I envision a new model for academic research and collaboration that breaks down the barriers dividing the disciplines. Building on the foundations of the Media Lab, I will create a vehicle for the exchange of ideas --- a vehicle that brings those working in antidisciplinary space together in exciting ways that challenge existing academic silos. My ultimate aim is to create a new platform and network for the 21st century: a new way of thinking and doing that will spread beyond the Media Lab, and beyond MIT.

Much of academia revolves around publishing research in prestigious, peer-reviewed journals. The peer review of academic papers was important in building scientific knowledge before the Internet, but in many ways it is holding us back now. It often leads researchers to focus on proving the value of their research to a small number of experts in their own field rather than risking an unconventional approach --- thus reinforcing a cliché of academia: ``learning more and more about less and less.'' And it exacerbates a hyperspecialization whereby people in different fields have trouble collaborating, even communicating, with one another. The Media Lab has just launched a new antidisciplinary journal with MIT Press, called The Journal of Design and Science, which is built on an open-access, open-review, rapid publication platform called PubPub, created by Media Lab students Travis Rich and Thariq Shihipar. As the curator of the new journal, I will work on creating a model of interaction online; many of the contributions will be snapshots of in-person conversations. This intimate form of communicating is in stark contrast to the formal peer-review system, allowing contributors to tackle the most interesting problems and ideas of our times; it is itself an experiment.

In addition to building this new way of collaboration and publishing, I would like to develop a new, hybrid research-and-development process --- a translation process that will deploy academic research into the real world and bring the real world into academic research. I intend to establish a program based on the original Media Lab design but serving as a link between MIT and the outside world. As participants/designers, we will be focusing on changing the way we do things in order to change the way the world does things. Thus I will recruit collaborators from across MIT and beyond. I have already launched a number of initiatives that are examples of this innovative outreach: 	

\begin{enumerate}
\item ``Extended Intelligence'' is the study of machine learning and AI as a human/machine system --- a fundamentally distributed phenomenon --- across scales from the neuronal/electrical, to interface design, to networks, to the formation of relevant policy and ethical standards for human/machine interaction.
	
\item The Media Lab Digital Currency Initiative enlists experts in computer science, cryptography, economics, and fiscal policy to work with a community of developers, companies, and government and public institutions on exploring the many issues involved in blockchain and bitcoin technology.
\end{enumerate}
	
The Age of Enlightenment centered on structured reason and heralded a new age of science. We are entering another new age --- one in which structured reason is not enough. A new kind of science is emerging, based on designing novel methods of addressing complex adaptive systems --- systems that remain beyond our ability to fully understand. My work will contribute to a new way of conducting antidisciplinary but rigorous research with global effects that will allow us to survive and flourish in this new age we have entered.
