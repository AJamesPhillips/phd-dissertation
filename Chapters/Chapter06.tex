\chapter{Conclusion}
\label{chap:conclusion}
The modern problems such as climate change, health, and societal inequities are complex and adaptive, and the only way to address them is through a paradigm shift away from reductionist market-based economic growth to a more sustainable and complex paradigm focused on flourishing. I have shown through examples that while computers and the Internet have added to complexity and speed, they also provide a way to redesign ecosystems and communities.

I explored the history of disciplines and used the example of the Media Lab to illustrate how the problem of silos created by disciplines may be addressed with an antidisciplinary approach, providing new academic communities with different values, funding, and reward structures. I have shown how this antidisciplinary approach can be applied effectively to tackle complex problems in many domains.

To tackle the world's wicked problems, I suggest a combination of the antidisciplinary approach and intervention in complex systems through a humble form of participant design and the transformation of values through social and cultural movements, using my days as a DJ, and my immersion in hippie and early Internet culture as an example.

I presented my ongoing work on transforming democracy and the public sphere and several new initiatives in space exploration, health care, and the governance and ethics of artificial intelligence as fields where I am testing my theory of change.

\section{Contributions}

This dissertation has addressed five central challenges: 

\begin{enumerate}
\item Draw on the history and philosophy of science and the learnings from operating the Media Lab to describe and explore the antidisciplinary approach and how it can be implemented effectively in an institution and a community.
\item Develop ideas from cybernetics, systems dynamics, evolutionary biology, and design as a way of understanding and intervening in complex adaptive systems through a method of cultural interventions in communities.
\item Show through examples how the values of decentralization and unbundling layers that emerged as the architecture for computing and the Internet can be applied to other domains such as finance, medicine, and climate.
\item Define the role of cultural movements in paradigm shifts and propose this as a way forward with the trifecta of wicked problems --- climate, health, and social inequality.
\item Define the importance of governance and ethics in the future of machine learning and artificial intelligence, and show how antidisciplinary scholarship and activity can advance the integration and appropriate deployment of machine learning and artificial intelligence in society.
\end{enumerate}

\section{The Learnings}

Throughout my practice, beginning with my experience working as a DJ in a nightclub and through my work on many layers of the Internet through for-profit and non-profit work, and cumulating in my current work at the Media Lab, there are consistent learnings.

\begin{enumerate}
\item The Internet has fundamentally changed the ability of communities to form and manage themselves at low cost while massively increasing the complexity of the environment.
\item Communities flourish with strong values that provide a shared mission beyond a purely financial one.
\item If the mission has a strong commons-based societally beneficial focus, the community can increase the resilience and flourishing of the broader ecosystem of communities.
\item Communities built around science, technology and innovation can be highly generative and creative with the right values and architecture.
\item The technical, legal, normative and financial architecture must be designed to provide a structure within which such communities can thrive, and for a community to appropriately interact with other communities in the broader ecosystem.
\item Leadership and design interventions in high functioning communities are participant-based, humble and have distinct differences from traditional top-down industrial firms.
\item An antidisciplinary approach can allow communities to transcend existing paradigms, and the Internet provides an opportunity to re-architect higher education and the development of knowledge and disciplines to support such an approach.
\item Complex scientific and social endeavors, such as designing Geiger counters and deploying them through workshops and kits, can be accomplished through a networked, mostly volunteer organization. 
\item Visionary leaders generate revolutionary new organizations and structures, but a more community-based leadership/management model is required to transform these organizations into sustainable and flourishing communities.
\end{enumerate}

\section{Future Work}

\begin{enumerate}
\item While the Media Lab has successfully developed an antidisciplinary approach to develop new disciplines and connect to existing ones, its rejection of structure has led to problems of structurelessness. Mentoring faculty in a system where each faculty member is unique and mentoring students in a structureless system is challenging. While the idea of learning through doing and collaborating to produce cross-disciplinary rigor does work, it is not a scalable system of knowledge. The consortium model of funding we use at the Media Lab has limited the negative effects of narrow funding sources, and our focus on research through making decreases the narrowing of peer-review for our own academic program. But we have yet to come up with new structures for knowledge, and can improve on the community's ability to manage the rigor and the quality of the work. The current Media Lab culture is generative and successful, but as we engage in harder sciences and try to interact with a broader community, additional structure and clarity may increase our effectiveness.
\item Mindfulness and the adoption of healthier and more personal motivators are key for the transformation of our values, but contemplative practice does not scale without losing many of its core attributes. Work on interventions that expand opportunities and incentives for contemplation, as well as ways to allow individuals to become more self-aware, can be developed.
\item The constructionist approach of the Media Lab helps theory through practice and practice through theory, but deploying that approach in the real world of criminal justice systems, health care systems, climate change advocacy, and other functioning systems goes beyond the scope of academic research. Creating nonprofit and for-profit spin-outs from the Lab, participating personally in non-academic operating entities disconnected with research at the Lab to avoid a conflict of interest issues, and formally partnering with operating entities through the technology licensing process have been the primary methods of deploying and expanding our practice and theory. Institutional development at MIT and similar institutions must be undertaken so that they can more actively engage for-profit and nonprofit startups to bring deployment closer to research.
\item Cryptocurrency and blockchain technology have many similarities and some key differences from the development of the layers of the Internet. But the for-profit focus and exuberance of that community is hampering the development of the non-profit and academic consensus and the protocol layer that is so essential to the Internet model. We must continue to apply what we have learned from the Internet to the evolution of blockchain technology.
\item While Creative Commons and the Open Source Initiative have contributed towards interoperability and reducing friction at the copyright layers, we still have many problems including a patent system that is stifling innovation by favoring large companies and by enabling frivolous patent filings and lawsuits. We are also challenged with the relationship between privacy, copyright and the use and abuse of data for both good and harm. While there are interesting conversations and new regulations, there is a great deal of technical and regulatory work required to solve intellectual property and data sharing on the Internet.
\end{enumerate}

\section{Call to Action}
\begin{enumerate}
\item With a combination of experiments to examine the human body; the development of new tools to interact with the human body, and the antidisciplinary application of knowledge from a variety of fields, we must re-imagine and redesign our understanding of, and ability to, intervene in our health system. This will require the creation of new research institutions and communities as well as the deployment of new business models.
\item We must advance beyond the ability to organize collective action through movements powered by the Internet and understand and deploy ways to scale the development of institutions, trust and collaboration --- a new democracy for the post-Internet era.
\item We must tackle the vital problem of climate change by creating a social movement with the tools we have developed for the Internet, drawing on learnings from historic arts and cultural movements.
\item We must shift from centralized, control-oriented design and engineering to participant design in all fields of endeavor and study, including space exploration, redesigning the financial system, and integrating machine learning and AI into society in fair and appropriate ways.
\item We must redesign academic publishing so that it becomes a viable platform and framework for sharing knowledge open and globally in the post-Internet, highly complex and antidisciplinary world.
\item We must develop and tweak the Safecast model of networked citizen engagement so that it can be used in other domains.
\end{enumerate}

\section{Summary of Chapters}

\emph{Chapter One: Introduction.} I present my introduction and an overview of the dissertation.

\emph{Chapter Two: Requiring Change.} I first explore the history of science and the creation of knowledge and disciplines. There is an historical view that disciplines are social and community-oriented, structured around power and community architecture more than ``truth.'' I describe the idea of antidisciplinary work between and beyond the disciplines and note that it is particularly important as the world has become more complex and faster thanks to the Internet. I suggest that we need to have shift our paradigms away from control-oriented interventions to participant design as the problems become too complex to control. I propose that influencing the values of a community might be a better way of causing paradigm shifts. I share my fear of the singularity movement as the next big reductionist movement and urge us to resist reduction. I present the hippie movement and its influence on the Internet as an example of a relationship between a cultural movement and a generative technical community.

I then outline five areas that I believe we can change. I describe the issue of silos that emerge from academic disciplines and how an antidisciplinary approach might be more effective. I describe how monolithic and centralized systems are being successfully unbundled; how the Internet was the first major success of unbundling, and how banking is now beginning to be unbundable thanks to blockchain technology. I describe the post-Internet public sphere and how social media has emerged in the context of the history of blogging. I then express my concern that we have used the Internet to dismantle institutions without using it to rebuild them robustly, thus undermining our faith in them. I worry that we have so far done more damage to the public sphere than we have done to help it, and that a similar situation could emerge for the future of health care: the pharmaceutical industry could fail before we have a viable alternative. I describe the pharmaceutical industry's struggle to keep up with the changing landscape of complexity and tremendous amounts of new data and tools. Climate is a similarly complex and potentially more disastrous problem that requires a fundamental change in the way we think as well as the way we intervene in it. We need social movements and participant design. It is all about community.

\emph{Chapter Three: Theory of Change.} I describe different types of people --- specialists, interdisciplinary people, multidisciplinary and antidisciplinary people --- and suggest that the role of the Media Lab and antidisciplinary people is to connect different disciplines and explore the spaces between and beyond them. I suggest that we might develop more structure, new values, and a new practice beyond just being antidisciplinary to create a new method for the development of knowledge.

I propose the idea that building of layers into the Internet's architecture, with commercial layers sandwiched between not-for-profit commons-oriented protocol layers, was the key to its success.

I give examples of the Digital Millennium Copyright Act, wiretapping laws, and anti-money laundering laws as laws that no longer work well and consider how we might use an antidisciplinary approach and Lessig's notion of the synthesis of law, markets, norms and technology in navigating these issues.

I share an essay on the nature of the Internet as a medium with aesthetics that influence its nature, art, and sensibilities.

I propose that activating communities is the key to paradigm shifts and share my experience as a DJ in a nightclub and the role that music had on various communities as an example of the role played by culture and sensibilities. I also use my experiences in nightclubs as a lens to see how the dynamics and reality of communities are nearly impossible to understand from the outside, and how working-class and bottom up values are important and underestimated by elites. I discuss the hippie movement and its methods and reflect on how we might consider the design of, and participation in, movements in the post-Internet era.

\emph{Chapter Four: The Practice of Change.} I draw on my experience and practice to illustrate the application of the theory of change.

The Internet and layers of the Internet with protocols such as \ac{TCP/IP} and \ac{HTML} have become wildly successful. The open source and free culture ecosystem has thrived in part because of the success of Creative Commons licenses and open source licenses verified and managed by the Open Source Initiative. Participating in the not-for-profit organizations that stewarded the protocols and the governance --- \ac{ICANN}, The Open Source Initiative (OSI), The Mozilla Foundation, Creative Commons, Computer Scientists for Social Responsibility (CPSR) and The Electronic Privacy Information Center (EPIC) --- helped me contribute to the success of the Internet's not-for-profit protocol layers as well as gaining an understanding of the ecosystem.

Helping build the first commercial Internet Service Provider in Japan, Infoseek, one of the first algorithmic search engines in Japan, helping launch the first banner ad network sold by impressions, helping to starting the first blogging software company in Japan, and helping to start the first and now largest payment settlements company in Japan provided an opportunity for me to contribute to, and learn from, the execution and scaling of the Internet and its services commercially.

Participating in numerous government committees helped guide regulation of the Internet, including laying the groundwork for independent ISPs in Japan, participating in writing the first hacker bill in Japan, and protesting, unsuccessfully, the national ID in Japan.

These activities substantially contributed to the development of the Internet ecosystem in Japan and the world. I learned a great deal about multi-stakeholder organization such as \ac{ICANN} and Creative Commons and how the communities can be managed to provide technical, normative and coordinating outputs. In addition to mission and leadership, such communities require rules, business models, and internal and external communication structures, which I describe in the implementation section.

Through publishing, blogging, and participating as a board member of \textit{The New York Times} and the Knight Foundation, I have explored and helped lead a conversation about the future of media and the public sphere. Our initial ideas were clearly optimistic about the nature of the transformation but accurate about the degree of impact. My efforts continue through work on the governance and ethics of artificial intelligence, through scholarship, deploying interventions, and teaching in order to improve the public sphere and the democratic governments that it serves.

Learnings from the Internet inform my management of the Media Lab which is also a generative, mission-driven community engaged in permission-less innovation. The Media Lab tackles the problem of siloing created by the disciplines through an antidisciplinary approach to scholarship and research. Together with the MIT Press, the Media Lab is tackling the future of antidisciplinary scholarship through reinventing academic journals and publishing.

Digital currencies and blockchain will potentially be another layer on the Internet stack and promise to be as transformative of law and finance as the Internet was to the media and commerce. I have described my work in advancing the field from the early days in the 1990s and 2000s testing early implementations such as Digicash, to supporting the early work on digital signatures and participating in an ambitious and failed attempt to build a data center, Havenco, beyond the reach of governments. I describe the creation of a substantial non-profit research effort in the form of the Digital Currency Initiative at MIT which hopes to help manage the protocol and community management required for digital currencies to flourish.

This antidisciplinary approach is being applied to understanding and designing interventions in the wicked problems of climate change, health and income disparity. Creating new sensibilities, communities, organizations, structures, scholarship and a new sensibility to improve our regulation and ability to flourish is essential. There is still a great deal to be done and each of these domains will necessarily be very different from each other, but many of the learnings and systems that I have helped create will contribute to our ability to improve our outcomes.

\emph{In Chapter Five: The Actors of Change.} I define happiness and differentiate between ``pleasure'', simple reward systems, and happiness through a sense of flourishing. I explore the history of interest-driven and constructionist learning for myself and the Media Lab. I discuss the role of cooperation and greed. I argue for the importance of disobedience in advancing science, law and institutions and the importance of disobedience robustness in organizations. I suggest ways that we might manage organizations to deal with not-so-beneficial troublemakers and trolls, and thoughts on the importance of self-awareness and humility in leaders.

\section{Looking Ahead}

The birth of the Internet and related technologies gave us hope that a new architecture would give us new values and a way to scale society up and out of many of the problems that faced us. It felt quite optimistic as many of our architectures and technologies revolutionized the way we did things.

As we face the reemergence of monopolies, polarization, and greedy feeding off of the commons on earth and possibly in space, many of us wonder if these social issues of humanity are pendulum-like, swinging from left to right, open to closed, optimistic to pessimistic over time --- or if they generally get worse with bumps along the way, or is true, as Martin Luther King said, paraphrasing the Unitarian minister and prominent abolitionist Theodore Parker, that ``The arc of the moral universe is long, but it bends toward justice \cite{parker1853justice}.''

That, I believe is for us to see and for us to cause.

\section{What's Next}

\subsection{The Internet}

We see that over and over again, the communities that grow up around new technologies and networks, mailing lists, Usenet, open source projects, email, and independent ISPs have difficulty retaining their core momentum and eventually break up through some combination of pressure from the outside; a flood of ``newbies'' dilutes and diverts the culture, or a layer of technology becomes irrelevant.

The ideas of openness, freedom, sharing and civility that we hoped to ``lock in'' when we designed the Internet were clearly subverted.

In addition, architectural elements such as the unbundled layers of interoperability and competition that the early ISPs exemplified have disappeared. Telephone companies and cable companies in the United States acquired or otherwise eliminated independent ISPs. Some countries still have them, but they are rare and not nearly the broad communities that they once were.

At the same time, the Internet has gone mobile, but there was no opportunity to unbundle the mobile system. As a result, we have a mostly metered and monopolized mobile Internet that costs far more than necessary, inhibiting basic activities such as roaming.

While activist organizations such as the Electronic Frontier Foundation and the American Civil Liberties Union continue to fight for the core principles of freedom on the Internet, it appears that the movement to keep the Internet open and free is diminished in power.

Even Apple, which initially provided us with the blinking cursor and a computer asking to be programmed, now makes it difficult and even scary to execute anything but officially approved applications. And ``jailbreaking'' your phone will break your warranty, if not land you in jail.

The world has gotten scarier and more dangerous and so it does make some sense that services are less open and generative, but the telecom and technology companies are clearly using this as an opportunity to lock in their control.

Although the recent issues with Cambridge Analytica and Facebook cannot be called ``bright,'' they have brought privacy to the fore of many citizen's concerns. The General Data Protection Regulation (GDPR) in Europe is bringing these issues into regulation. Ironically, it may turn out to be government that takes the lead into the next phase of Internet civilization.

\subsection{Blockchain}

At the Media Lab, we have convened a group of strong researchers and academics from diverse and relevant backgrounds to work on the issues raised by blockchain technology. The core members of our group do not have financial interests in cryptocurrencies and are not biased financially. We are clearly playing an important role by making public comments, advising governments, and thinking long term about the development of the blockchain for social good.

While I theorize that the blockchain will go through a development pattern similar to that of the Internet, with layers, interoperability, non-profit non-governmental coordinating bodies, and layers of venture businesses in between, there are some clear differences.

We do not have identifiable community leaders like Jun Murai, John Postel, Richard Clark, Vint Cerf, Steve Crocker, Tim Berners-Lee and others who can take the lead in bringing communities around blockchain together. Cryptocurrencies originally had an anti-government culture that it is now growing out of but still has its roots in, whereas the Internet was quite government-friendly for the most part. The blockchain is stateful and technically quite different from a communications layer. Lastly, but perhaps most importantly, the financial impact of the blockchain has eliminated at least a decade of time, if not more, for the technical and hobby people to play with standards and technologies before trying to build businesses on top of it.

I still believe that Bitcoin has the most substantial community, but it is missing some coordinating ability, a compelling culture and leadership.

Regardless of these caveats, I believe that it is critical to try to develop a community around Bitcoin and support its development using all of the tools and the methods so far discussed.

\subsection{The Commons}

Creative Commons licenses have continued to grow, and they are now an important part of the publishing ecosystem. We have not, however, solved the problems of open access to knowledge and academic publishing.

Free and open source software have become a standard and valuable part of the ecosystem, but their use and management can still be improved. In fact, the fact that Bitcoin is an open source project is a significant contributor to its success, and in the field of cryptography where peer review is essential, it is nearly impossible to have a non-open source project.

Patents are still a major problem for small companies that compete with large companies. Individual researchers are also victim in many cases to large company patents. The patent system clearly needs an overhaul.

Both the open source community and Creative Commons have tried to address the problem, but we may require government, academia and businesses to work together to resolve this.

\subsection{Health}

We are just embarking on the implementation of what we have learned about communities, breaking up the silos, and creating new disciplines, and the response from researchers, regulators and businesses has been exciting. The initial workshops and meetings have been productive, and the active participation of the U.S. Food and Drug Administration is heartening.

The complexity of the problem and the number of pieces that we must manage is daunting. I am also concerned that, as with media and the Internet, it is possible that we destroy the old model before we have come up with a new one.

One impediment is the difficulty of integrating commercial ventures and academic scholarship because of conflict of interest policies put in place to protect the neutrality of academic research. Designing better ways to translate research from academic research labs into commercial deployment has a great deal of opportunity to improve.

\subsection{Space}

The Space Exploration Initiative, like the Health 0.0 project, is also an emerging project that addresses a very diverse community, tying in to existing efforts at MIT in the Department of Earth and Planetary Sciences and the Department of Aeronautics and Astronautics. The event this year for the initiative was the most watched event in the history of the Media Lab online.

Danielle Wood, a new faculty member of the Media Lab, pointed out during the event that we should avoid the term ``colonizing space'' since the word ``colonizing'' has its roots in extractive, exploitative conduct by the West. We should be more reflective and respectful of space. The issue of whether we have the right to treat space as something we can own and what our relationship to space should be has many parallels with the way we have treated earth. We are already beginning to understand the damage we have caused by strewing space junk in our orbital space around the earth.

The issue of the law of space has many similarities to the ``laws of cyberspace,'' where we try to understand how to manage ourselves in this extra-jurisdictional space and protect the commons.

The diminishing cost of participating in space exploration has some parallels to the effect that the Internet has had on many fields. The the initiative and the diversity and number of projects proposed for it show how similar it is to the development of the Internet. The application of Internet learnings on space will be an interesting area of exploration.

\subsection{The Environment}

\subsubsection{Nia Tero}

Nia Tero strives to protect the climate through the protection of indigenous people. At a recent meeting, we discussed the notion of ``natural capital.'' This is the idea that if we account for the carbon, the trees, and the fish, and track the ``assets'' that companies and society are extracting from the environment, it would help us account for these externalized costs and manage companies and natural resources better. One of the indigenous leaders pointed out that to him, the term ``natural capital'' sounded like an oxymoron. Nature does not belong to us. We belong to nature. The idea that nature should be in our accounting systems might make sense if we are trying to tackle the climate and environmental issues using the market and ``enlightened best interest'' to solve the climate problem, but the true paradigm shift would be to adopt the sensibility of indigenous people and make it feel ``wrong'' to everyone to just take from nature.

At a recent meeting in Hawaii on Earth Day 2018, I was involved in a discussion about the Rodium Group's ``Transcending Oil: Hawaii’s Path to a Clean Energy Economy'' report \cite{Transcen68}. The report shows that Hawaii can be completely free from fossil fuels by converting to renewable energy. The report lays out a clear argument that the economics work. As we discussed the issues, it became clear to me that what we were talking about was the transformation of existing institutions and the transcendence of existing paradigms developed around a notion of centralized control of a fossil fuel-based energy resource and supply chain. The decentralized and ``out-of-control'' idea of a power grid that connected everyone with solar energy generators was a completely different paradigm, but that paradigm, at this point, is a greater challenge than the technology.

Interestingly, the behavior of energy utilities is extremely similar to the behavior of telephone monopolies during the development of the Internet, and it appears that the challenges and benefits of decentralization and unbundling will be very similar.

\subsubsection{Safecast}

We learned from the Safecast experiment that we can mobilize engaged scientists and citizen activists by engaging them in a movement. The Safecast movement involved social activism and active participation in the building, deployment, use and sharing of the system. The open data platform was key. Safecast has emerged as an extremely successful model, and we are being pulled into new areas of bottom-up participatory science in adjacent spaces such as air quality.

\subsection{Media Lab}

As the Media Lab enters its fourth decade and begins working in the hard sciences and grappling with the complex problems of health, climate and social inequity, we must reposition it among the disciplines. The Lab also has become less constrained financially, and we must now consider the limits of growth and the way in which we select our domains and define limits.

More work in the hard sciences also increases the necessity for a more structured system of knowledge development beyond the antidisciplinary.

\section{Concluding the Conclusion}

The future of humanity depends on our ability to understand and intervene in society in order to minimize the impact of climate change: halt and reverse damage from social inequity; stop the march of chronic disease, and consider new ways of thinking about health; maximize the benefits of new technologies while minimizing their detriments; and explore new areas such as space and the blockchain.

We can apply the ways we unbundled power and created layers to organize and build the Internet to many cases that analogs, even if they don't work exactly as the Internet does. 

Understanding the systems and levers of intervention using Lessig's laws, markets, technical architecture, and norms is key. (See \autoref{fig:lessigian} for diagram of Lessig's quadrants.) An antidisciplinary approach and community are key. Donella Meadow's notion of intervening at the paradigm level is key \cite{meadows_leverage}.

The creation of this capacity, as well as the ability to manage the change that it will produce, requires a healthy and well-designed community and movement. The Media Lab can experiment with community design and management itself, as well as new initiatives being developed at the Lab. Hopefully the Media Lab and the projects and the people who pass through it will serve as inspirations to other to use, adapt, transform, and share our ideas and methods.

Supporting the Media Lab in these efforts will be the focus of my work for the foreseeable future.