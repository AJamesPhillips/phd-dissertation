% Appendix E

\chapter{Joi and the Internet in Japan}
\label{chap:jefu}

\textit{by Jeffrey Shapard} \\
\textit{April 2018} \\

This is a personal perspective about how my longtime inspiration, colleague, and friend Joichi Ito influenced the development of the Internet in Japan. It is written in the third-person to create the aura of objectivity, but this is a subjective story. Most of the events and dates and names are true, or close enough, or at least as much as the author can recall in his advancing age and late hour writing. 


\textbf{How it Began: The Seed of Inspiration (1983 or so)}

Back in 1983 or so, Joi was a high school student at the American School in Japan (ASIJ) in western Tokyo and really liked computers. He was in the computer club at his high school. He was active on some early online systems in the US and he was a bit of a hacker, in the exploratory rather than destructive sense. One day he hosted the monthly meeting of the RINGO Club, a user group mostly older expat Americans who had Apple II personal computers, to show them what they could do with an Apple II and an acoustic coupler that let them connect via telephone lines to other computers far away.

Joi fired up his acoustic coupler, dialed a bunch of numbers into his phone, jammed the handset into the coupler, and proceeded to give a worldwide online tour. He started by logging into a couple American online services, CompuServe and The Source, the big boys of the 1980s, to read some forum comments and post a few of his own. Then he logged into a university computer in the UK that hosted a multi-user game he enjoyed. After that he connected out from that university site to hop through a couple more and found a backdoor into some government site, just to show that he could. 

All simple stuff for a young growing up with computers, but it opened up a new world for the older RINGO Club members. One of those RINGO Club members was David G Fisher, a retired US Air Force pilot, longtime resident of Japan, and business and English language educator at the International Education Center in Tokyo and its English language school Nichibei Kaiwa Gakuin. Mr Fisher also liked these new personal computers, and he was always looking for ways to use them in education.

Joi's presentation of online systems stunned him, both in how easy he made it look as well as in the way it opened up the world. Imagine participating in a conversation with people on the other side of the world who were sleeping, but could read your comments and respond later when they woke up. He immediately saw the potential for educational applications.

Mr Fisher had a private student and friend named Toshiaki Tanaka who was president of a shrimp wholesaler and seafood retailer called Sakako Co. Ltd. Tanaka-san also liked personal computers, and was interested in their use for communication and for education. He was using them in his business to consolidate data and to connect his shops, and had hired a very bright game programmer named Makoto Ezure to help set things up. In those days, every Japanese personal computer maker had their own flavor of mostly CP/M-86 operating systems, competing kanji character codes, and very little software. So, Japanese computer users would select their preferred maker and then write their own software. Tanaka-san liked Fujitsu computers and Ezure-san wrote the software for them.

Meanwhile, back at the school where Mr Fisher worked, there had been a drop in attendance so reduced teaching hours were available. The director of the school that specialized in training for business people asked a younger teacher named Jeffrey Shapard if he was willing to do research on educational technology for a couple terms rather than teaching, and he accepted. The director instructed him to hang out with Mr Fisher to get ideas. And so he did.

After a review of various educational software and applications available, and much discussion inspired by what Joi had shown Mr Fisher, the teachers determined that test-taking and quiz applications were far less interesting for language learning that using language to communicate via computer. Mr Fisher then brought Jeffrey to meet Tanaka-san and Ezure-san, and the plan for TWICS as an online system was born.

The plan was presented to the administration of IEC/Nichibei, who agreed to enter into a joint venture with Tanaka-san. Sakako would provide the computers and Ezure-san. IEC/Nichibei would provide Jeffrey and Mr Fisher. And any decisions about eventual commercialization would be joint. The initial purpose was to build a platform for distance learning via online communications.

And the inspiration for this, as Mr Fisher often stated, came from very young Joi, who around that time was graduating from high school and moving off to America to go to college.

\textbf{Pre-Internet days: TWICS (1984-1993)}

In the early 1980s, there were no online services in Japan. By law nobody but the telecom monopoly was allowed to connect anything to the telephone line. There were no modular jacks or handset variations not provided by the monopoly. The only legal way to connect a computer to another via phone line was to use an analog acoustic coupler device, where you had to dial the number by hand, then mash the handset down into phone pads to transmit the relatively slow signal.

There was one personal computer-based bulletin board system (BBS) called CortNet, run by an American named Pete Perkins who operated a computer shop in the Sanno Hotel, which was a US military R\&R facility in central Tokyo. So, legally, those phone lines were US territory, and the US had just recently changed telecom laws to allow individuals to connect devices to the telephone network. CortNet had a modem device that answered the phone automatically, although to get to them one had to dial the Sanno Hotel reception desk and then request a transfer to the computer shop to get the modem to answer. It was cumbersome, really rattled the reception desk staff, but was totally amazing to first time online users.

Tanaka-san told Ezure-san not the break the law, so in addition to building the BBS-inspired software application for the Fujitsu personal computer platform, Ezure-san also had to fabricate a Rube Goldberg device involving acoustic couplers and a controller attached to wiring in the phone (not to the line) to answer and hang up calls. The first TWICS online system launched in 1984, hosted by 2 networked Fujitsu personal computers and 2 acoustic couplers for two simultaneous online users. It was then either the first or second BBS in Japan other than CortNet, and depending on claims about who went live earlier in one 24-hour period. The other BBS was set up by a local expat PC user club using computers, software, and modems imported from the US, the latter illegal by Japanese law, and for the purpose of talking tech and swapping bootleg software.

TWICS was named by Tanaka-san, who was bearing most of the costs. He wanted it to mean “Two Way Information and Communication System”, although when his accountant registered the name and rendered it into a Japanese pronunciation, it got really mangled. And Tanaka-san operated with the brand Honeymoon for his shrimp, and bees make honey, and honey makes people happy, so he wanted our logo and motif to involve bees. Hence, the official full name TWICS BeeLine. It all made sense at the time.

And then Joi came home from school for the summer and joined TWICS. Jeffrey had been in email communication with him via The Source and they had been exploring how to set things up for TWICS to be a more relevant platform than a BBS, with communications more meaningful than shallow chatter, and with a way for groups, classes, teams, communities to evolve online. There seemed to be no terms that captured what they were doing: pasocom tsuushin (PC communications), going online, computer conferencing, electronic networking. The few people doing it in those days understood it, but it was very difficult to explain to others, and therefore a challenge to promote and grow.

Joi became an advisor and member of the TWICS team. He had experience on various other systems, and helped get conversations and momentum going on that first system that got beyond the shallow chatter of most of the BBS world of that era. And he helped the team understand that it is not necessarily about the technical platform, but more about how it is applied. It is all about the application and the people, not the technology.

Then Joi went back to America for school for a while. In 1985 TWICS outgrew 2 phone lines and, in a crash development marathon, built and launched a new software platform on Unix-based host computer, with new applications for email and the forums, and fancy new higher speed modems connected directly to the phone lines. The laws had started to change and TWICS expanded to 4 lines. However, the host computer proved inadequate for the task and the team was distracted by having to build and support all the software, which distracted from the focus on communications, groups, classes, and the emerging virtual community.

Then Joi came back to Japan again, this time to stay longer. He advised the TWICS team to get out of the software writing business, obtain some decent software for a reliable platform, and focus on the application and service, not the technology. And so they did.

After a survey of what was available, they selected PARTICIPATE, the software application for computer conferencing used on the big American online service called The Source, and a DEC MicroVAX II running the VMS operating system to support it. This time the software drove the hardware decision. In addition to the 6 phone lines, the TWICS system was connected to a couple X.25 networks as alternative to long-distance dialing for users beyond Tokyo, and beyond Japan. TWICS went global.

By then Jeffrey had become the system administrator. He challenged Joi to hack into the system. He tried, and would eventually have succeeded, so Jeffrey gave him full root administrator privileges because he would have figured out how to get them anyway.

Joi became an active member of the TWICS online community and inspired others. Joi started a monthly meeting for people interested in telecommunications with personal computers called T-Net, where he often presented telecom tricks and techniques, or gave tours and demonstrated things on other systems, and generally shared his knowledge. Many TWICS members came to T-Net and what became the traditional lunch and Saturday afternoon party at an Indonesian restaurant down the street. Some of those participants in the T-Net meetings went on to start other online services or spread the use of the technology into the businesses and organizations.

This combination of a virtual community with regular opportunities to actually meet other members face to face became a critical feature of the community, and more important as Joi invited in other folks he know from his global networking travels to become guest members of TWICS. Eventually there were members coming in from 25+ countries before the Internet made it all so much easier and cheaper, and it was common to have international visitors drop by for the monthly meetings. This created a unique educational opportunity for the Japanese members, and created context for real experience using English in real communications with real people.

But the Internet was coming. In 1990 or so, Jun Murai, the founder of the Japanese academic network JUNet, gave a presentation to the International Computer Association and said that their academic network was open to commercial entities, if interested. Jeffrey pounced and asked Murai-san to let TWICS connect to JUNet. Within a couple weeks, TWICS had an Internet domain name (twics.co.jp) and a dialup uucp connection to a JUNet host computer for distributed email and news groups.

But international telecom rules in Japan made it difficult for Internet connectivity beyond Japan for anything but academic purposes. Meanwhile across the Pacific in the US, the Internet had escaped academia and begun commercial development.


\textbf{Intro of the Internet: IIKK (1993)}

Jeffrey left Japan in 1992 and returned to the US for business school. He was contracted by a US software firm called Intercon during the summer of 1993 to go back to Japan and help their Japanese joint venture partner set up an Internet access business. They had one commercial customer already waiting for them.

After several weeks trying to gain access to the old karaoke bar that was to be used for the office of the new Internet venture, in the final week Jeffrey and his small start-up team of his wife Masaji and classmate Bill Hodgson got the new company incorporated, got license approval from the Ministry of Posts \& Telecoms, and recruited a local manager and staff. The company was registered as Intercon International KK, with internet domain “iikk.co.jp". The company had also managed to register the domain “inter.net”. For various reasons, nobody else had. 

However, before the network service could be set up and made operational, an ownership dispute between Intercon and their local joint venture partner got the start-up team kicked out of their karaoke bar office.

Jeffrey went to Joi for ideas, and Joi found IIKK a room for an office and a spare bathroom in a condo next to where he had his office.

A dedicated international circuit for Internet access service was pulled into the spare bathroom, and a router and server were installed in a small rack in the shower stall for the access point. The first commercial Internet access point in Japan was in a shower stall! An ethernet cable was strung out the window and over into Joi's window for his Internet access, and his office became the first non-academic site in Japan connected to the international Internet when the service went live.

The first commercial customer of IIKK was TWICS, whose local community was one of the first in Japan connected directly to the Internet, and whose service rapidly expanded as one of the first dial-up Internet access providers.

Jeffrey left Tokyo and returned to the US and Joi continued to support the fledging Internet access business.

At the end of 1993, Intercon decided to sell IIKK to PSINet, one of the pioneers in the commercial Internet business in the US. Jeffrey returned to Japan with Bill Schrader, the president of PSINet, for due diligence before the final decision and Joi was the first person he met in Japan the night he arrived. PSINet acquired IIKK, fired the local manager, and put in place a new team that would become PSINet Japan.


\textbf{Building the Internet: PSINet Japan (1994-2000)}

Joi continued to provide moral and opportunity support to the local technical employees Vince Gebes and Eric Bowles and a series of American managers. The business grew slowly due to marketing and sales constraints, but Vince began to emerge as team leader. However, he was an engineer and not yet ready to build a business.

In the meantime, PSINet went public in the US in 1995 and began to expand their network infrastructure and business operations rapidly across the US and Canada, and made a new acquisition in the UK. In 1996 they turned their attention back to PSINet Japan.

The PSINet executive team wanted a seasoned ``industry professional'' to build the company, even though the Internet was yet a nascent industry with a business model that would basically disrupt all the established industries it touched. They thought they wanted an old school manager but they needed an agent of change.

So, despite the objections of some other executives, PSINet CEO Bill Schrader called Joi and asked him to lead PSINet Japan as President. He knew that Joi had another business, various other projects in motion, maybe a board or official committee seat or two and various other time conflicts, and was not much of an operational manager. But he was a player not afraid to be bold, he quickly understood the ambition in the PSINet ambition, and he saw how everything was connected.

Joi was interested in everything, he knew everybody, he was enthusiastic, and he agreed to help PSINet get to the next level in Japan. And this time he actually got paid for his support.

Joi moved the company from their crowded condo space to a nice office in Akasaka, got the team all excited about fast growth, and hired a party space for a big coming out party. He invited the shakers and movers in the technology, media and emerging Internet businesses, then followed through with a public relations blitz and media interviews. He rebuilt critical relationships that had been neglected, initiated new ones needed to move forward, and promoted the Internet everywhere.

Basically, Joi put PSINet on the map in Japan. Then he handed off ongoing marketing and business growth to Vince and their growing team, and departed for other ventures.

PSINet Japan continued on the trajectory set by Joi, with bold marketing, impact beyond their size, and a willingness to try new things. Vince became the next president, and within a few years as PSINet raised even more money, they acquired three more Internet companies in Japan, including TWICS where it all began, and became the headquarters for PSINet in the Asia Pacific region where numerous other Internet service providers were acquired in other countries. It was a time of great expansion.

Alas, one day investors woke up caring more about profitability than expansion in the now booming but overheated Internet industry, and by 2000 PSINet had crashed and burned and was sold off in pieces. PSINet Japan operations were acquired and integrated into another telecoms company in Japan.

Vince and Eric and other colleagues from PSINet Japan, including Tim Burress from TWICS, went on to found a successful Internet security services business.

\textbf{From a Seed to the Internet}

Had Joi not shared his early online experience as a teenager with a bunch of early Apple II hobbyists in Tokyo in 1983 or so, there would have been no TWICS and IIKK and early PSINet Japan. The Internet would still have emerged in Japan without pioneering Joi and his American friends, but more slowly and perhaps in a more insular and regulated manner.

Joi did not specifically build the day-to-day operations of TWICS or even PSINet Japan, but was somehow always there at key times to influence development or give things a nudge in some new direction, whether as a community member, a friend and free consultant, an advisor to the board, or as company president. Or maybe those times became key because he was there.

Joi influenced the Internet in Japan by connecting people and ideas early and significantly, by asking creative questions, and by constantly promoting a vision of the potential of trying new things and new ways.

