% Appendix D

\chapter{HavenCo, Ltd. Business Plan}
\label{ch:havenco}

Non-Disclosure: The HavenCo business plan is confidential.  Neither the plan nor any of the information contained herein should be reproduced or disclosed to any person without the written permission of HavenCo.

Disclosure Regarding Forward-Looking Statements: Some of the information provided in the HavenCo Business Plan may contain projections or other forward-looking statements regarding future events or the future financial performance of the Company. We wish to caution you that these statements are only predictions and those actual events or results may differ materially.

Disclosure Regarding Offering: Investment in HavenCo is highly speculative high-risk venture. Only those who can afford to lose their entire investment should respond to the offering. Local laws may affect an investor's ability to participate. Before proceeding with an investment in HavenCo, any potential investor should take care to become familiar with local laws governing investment in foreign corporations, especially any associated reporting requirements, and taxation rules that may apply.

\textbf{Executive Summary}
 
HavenCo, Ltd. is exploiting a unique opportunity to set up the world's first real data haven. The target location is the Principality of Sealand, the world's smallest sovereign territory. HavenCo is building a secure managed co-location business with the added advantage that the customers' data will also be physically secure against any legal actions. Since the co-location business model is a generally profitable one, we expect to continue to be profitable at that site even if a larger nation manages to force some level of regulation over Sealand. HavenCo also intends to use the Sealand operation as a model to demonstrate the profitability of zero-regulation e-commerce to other small countries around the world. We will then be able to eliminate any single point of failure by replicating the "haven" in other jurisdictions. This will also reduce the visibility of our initial showcase site, which will continue to have the best connectivity. HavenCo is currently seeking up to \$3,000,000 in first round funding to establish its showcase data center and begin servicing new customers.

\textbf{PROBLEM / OPPORTUNITY}

The countries that currently have the best infrastructure for e-commerce are suppressing the growth of profitable Internet business through prohibition and regulation of content. Any company that can offer hosting services in a jurisdiction that both allowed free and private data communication and has access to first world bandwidth will have a unique and highly profitable business.


\begin{table}[h]
\centering
\caption[Havenco business plan chart of quality and law-enforcement trade-offs]{Businesses engaged in electronic commerce currently make a fundamental choice to operate from the first-world or the third-world with the following trade-offs.}
\label{havencotable}
\begin{tabular}{lll}
\hline
               & \textbf{First-World}      & \textbf{Third-World}         \\ \hline
Infrastructure & High Quality / Low Cost   & Low Quality / High Cost      \\ \hline
Regulation     & Random / High Enforcement & Negotiable / Low Enforcement \\ \hline
Taxation       & High / High Enforcement   & Negotiable / Low Enforcement \\ \hline
\end{tabular}
\end{table}

It is very difficult, if not impossible, to run businesses which require very high reliability, high-quality infrastructure without regulatory hindrances. Businesses that require high quality e-commerce infrastructure face a significant burden in costs imposed by taxation and regulatory compliance. This prevents many businesses from forming in the first place, and limits the chances of success for those that do start up.

HavenCo will answer the infrastructure vs. freedom question in a fundamentally new way, applying novel technology, a unique physical location, and a world-class team.  We will provide business with better quality infrastructure than ever before, allowing eCommerce operations the luxury of an environment free of unnecessary regulation and taxation, and at a lower total costs than anywhere else.

HavenCo intends to target specifically:

 
\begin{itemize}
\item transaction-oriented businesses, such as electronic gaming, financial and securities systems, and critical business infrastructure such as Application Service Providers (ASPs);
\item security-dependent businesses, such as certificate authorities, records archiving, and security infrastructure businesses;
\item network-centric information-processing businesses (e.g. music, software, graphics, streaming video content, and network infrastructure such as outsourced mail, news, web servers)
\end{itemize}

These market segments fit very well with our potential product and service offerings, and will bring high profitability and rapid growth.  Businesses in these markets face the greatest dilemma in selecting between first-world and third-world infrastructure support. The market is enormous, and growing rapidly, with no competitor providing products and services that simultaneously fulfill all of these customer's needs.

Critical requirements in our target market segments include security (confidentiality, integrity, and availability), transactional performance (primarily driven by latency to the end-user), and ease of use (support existing operating systems and applications).

In order to meet these requirements we will employ several cutting edge or novel technologies.  These include:

\begin{itemize}
\item Ultra-high bandwidth IP communications directly into the Internet backbone (STM-1 to STM-16 and higher), and gigabit-speed internal networks, with superior routing and management
\item Fully redundant power, cooling, network, and management systems, using 2N redundancy when possible
\item Tamper-resistant computing hardware, designed to protect customer transactions from all possible attackers, including HavenCo and its staff
\item Advanced cryptographic protocols to support access control, financial transactions, and secure transaction backup 
\item Open-source software modifications to allow customers to use existing, reliable, well-understood software while exploiting the features of tamper-resistant and cryptographically-secured servers
\end{itemize}

In order to maximize profitability, HavenCo is designing for maximum density, minimum maintenance requirements, believing that a good design and quality equipment will more than pay for itself in reduced labor and overhead and improved quality of service.

In addition to the technologies we will implement and develop to support our core collocation business, HavenCo will be able to use advanced technologies in combination with our unique regulatory situation to offer value-added services never before seen. Such advanced projects will likely include internet-based equities markets and cryptographic token-based payment systems. HavenCo may develop this technology, or partner with others who can already supply it. We will be in an ideal position to market these additional services to our existing customers, and will be able to use them internally as well. It is via such technology that an eventual Internet Public Offering of HavenCo is expected to take place.


\textbf{LOCATION}

A unique asset to HavenCo is the location of its initial showcase data center - the Principality of Sealand. Sealand is the world's smallest sovereign territory. It was founded over thirty years ago, and has obtained a unique legal status as the only sovereign man-made island. Its claim to sovereignty has been tested and supported in several legal challenges.  (See included report on the history and current legal standing of the Principality of Sealand.)

HavenCo does not completely depend upon the continued legal status of Sealand as a de-facto sovereign nation, but is in a position to profit substantially from that status in conjunction with a first-world location. Sealand is located less than 3 milliseconds (by light over fiber) from London, home to leaders in both global finance and international telecommunications. Other than San Jose, California, London is perhaps the world's premier Internet exchange point. Sealand has no laws governing data traffic, and the terms of HavenCo's agreement with Sealand provide that none shall ever be enacted.

In the event that some other nation should manage to successfully exert its jurisdiction over Sealand, the location will continue to provide unique advantages. The legal fight surrounding a challenge o Sealand's sovereignty will provide for a great deal of publicity. If forced to capitulate to a larger nation, it should be possible to leverage Sealand's history and publicity into special status for Sealand. Britain, Sealand's nearest neighbor, is the only real threat in this regard.  It already has many territories with special status and exemptions from many of its laws.

Even if Britain successfully obtained complete control, Sealand would continue to remain a viable location for a secure co-location business. Co-location is a very profitable business model, and we would enter the rapidly expanding market amid a great deal of publicity and attention. This publicity and attention should point to the profitability of our business model, and help us in our plan to replicate the zero-regulation eComerce environment elsewhere.

\textbf{REPLICATION}

The establishment of such the first zero-regulation e-commerce jurisdiction may provoke renewed challenges to Sealand's status. HavenCo therefore plans to replicate this situation as soon as possible at an independent location. Regulatory concerns aside, engineering for redundancy is good systems design policy, and many customers will pay for redundant servers in widely separated physical locations.

Using the publicity and revenues obtained via our Sealand location, we will approach those small governments that are only now just beginning to receive major connectivity to the Internet. The possibility of getting a share of the widely publicized e-commerce marketplace, combined with our demonstration of a working model, should be enough to convince such small governments to establish e-commerce free zones in their countries. Once this begins to happen, our Sealand location will become less unique, and therefore less prone to challenge.

\textbf{TEAM}
    
The HavenCo founders, initial investors, and management consist of experts and visionaries from the network infrastructure, security, and e-commerce industries. Additionally, in spite of the tight global market for technology labor, sufficient staff has already been identified for the first year's operations. We are assisted in filling staffing requirements by both the low manpower needs of the high-density, low-maintenance philosophy, and the fact that our business model holds unique ideological appeal for a fairly large segment of the technology aware community.

In addition to the core team, many vendors are actively participating in a ``build to order'' and financing role, greatly assisting HavenCo in its mission.

HavenCo's founders and core team members include:

\begin{itemize}
\item Sean Hastings --- Chief Executive Officer of HavenCo. Sean was previously CEO of Isle Byte Inc, a Caribbean based consulting company specializing in the development and implementation of hardware and software solutions for telephone and Internet based businesses in offshore jurisdictions. Isle Byte's recent projects have included: the Phone-Book touch-tone telephone and Internet sports betting system for offshore sportsbooks; design of the HOB protocol for SAXAS, an account based eCurrency system being developed by Secure Accounts Ltd, a Caribbean based financial software company; and software development work for Domain Marketplace, a domain registration company for a pacific island Top Level Domain.
\item Jo Hastings --- Chief Marketing Officer of HavenCo. Previously Jo worked as Program Manager for Isle Byte Inc specializing in the set up and operations of Internet casinos from both the Anguilla and California offices. Prior to Isle Byte, she was a Senior Market Analyst for Urban Systems Inc in New Orleans where she edited and wrote for the Gulf South Gaming News, consulted for the Gaming Industry Research Institute of the South and was published in Casino Executive magazine. Her specialty was tribal casinos in the United States as well as emerging technologies for casinos, such as the Internet. She currently sits on the Board of Directors of the Crypto Rights Foundation based in San Francisco.
\item Ryan Lackey --- Chief Technical Officer of HavenCo. Ryan Lackey has worked to bring high security, pro-individual-freedom technologies to the marketplace, first while a student at MIT, and then later in startups developing cryptographic electronic cash solutions for a variety of markets.  Ryan has presented at several conferences and symposia in the security field, and is well known within the security community.
Avi Freedman - Chief Network Architect of HavenCo, also currently VP of Network Architecture for Akamai. Previously he was VP of Engineering for AboveNet. He founded and continues to oversee operations of Netaxs, the first ISP in Philadelphia, founded in 1992. Freedman is also a regular contributor to Boardwatch.
\end{itemize}
 

HavenCo's team of advisors include:

\begin{itemize}
\item Sameer Parekh --- Consultant. Sameer is the well-known founder of C2Net Software, Inc. the leading provider of commercial Apache products and solutions. As CEO of C2Net he pioneered the international offshore cryptography development strategy later adopted by RSA Security in order to deploy strong cryptography worldwide in the face of United States restrictions on the export of strong cryptography. C2Net currently has leading market share in the encrypting web server arena. Sameer is currently a consultant at his own practice known as BPM Consulting International, helping young seed stage startups develop themselves.
\item Joichi Ito --- President of Neotony, a Japanese Internet startup company incubator. Founder of Eccosys, Digital Garage, and InfoSeek Japan. Technical advisor to the Inter-Pacific Bar Association working on a cyber arbitration project. Working with UNCITRAL on rules for arbitration in cyberspace. Working on a committee concerning Japan's position with the WTO and the resolution of transborder legal issues.
\end{itemize}

\textbf{OPERATIONS TO DATE}
 
HavenCo has made substantial progress toward accomplishing its plan.  Since June 1999, HavenCo has done the following:

\begin{itemize}
\item Discovered the possibilities offered by Sealand and made contact with the owners;
\item Visited the Sealand site and inspected it for feasibility of use as a data center;
\item Secured a lease with option to purchase on the \item Sealand facility with highly favorable terms;
\item Assembled a team of experts from the infrastructure and major client industries to conduct operations;
\item Attracted a feature article in the August 2000 issue of Wired magazine, whose editor has said that our story is a good contender for the cover;
\item Developed a technical plan with vendor cooperation to implement a world-class data center at the Sealand facility while requiring minimal capital;
\item Identified and pursued key technologies that support high-quality, highly secure infrastructure;
\item Located several initial sales leads;
\item Concluded agreement for our first pre-sale.
\end{itemize}

\textbf{MILESTONES}

June 1999: Discover Sealand opportunity and begin planning, negotiations, and team formation

November 1999: First site visit and inspection

February 2000: Complete lease/purchase agreements on Sealand; begin accepting investment

March 2000: Conduct engineering tests, establish IP connectivity, local network, initial servers, and begin limited presence on Sealand. Establish London Telehouse presence, routers, transit and peering, and colocation space.

May 2000: Develop sales and marketing materials

June 2000: Pre-position at least 50 servers with at least 45mbps of primary backbone connectivity with backup connectivity.  Target is 100 servers with redundant 155mbps connectivity. Sell at least 10 machines to key customers prior to launch, under nondisclosure, to debug sales and support.

July 2000: Public launch --- publication in Wired, followed by extensive press coverage in general and specialist publications.

August 2000: Seek round two financing from a major network hardware vendor if necessary.

September 2000: Identify possible sites for replication and begin to negotiate with governments in those countries for favorable terms in setting up e-commerce free-zones.  

December 2000: Unit sales of at least 25 major customers.

April 2001: Sign agreement with replication site \#1 and begin construction of second data center. Begin taking pre-sales orders for redundant machine location from current and new customers.

July 2001: Meet sales target for break-even (50 major customers)

December 2001: Internet direct public stock offering

\textbf{PROJECTIONS AND FINANCIAL ANALYSIS}

Given the rapidly-expanding market for transaction servers on the Internet and the modular and infinitely extensible technical plan, HavenCo projects depend critically upon market share. Consequently, we have chosen to pursue the model where we rapidly build market share by offering a superior product at dramatically lower prices than others. However, compared to most ``Internet Businesses,'' we can retain mid-range profitability while following this model, and to the extent required, margins can be cut to meet sales targets, either by reducing costs or (preferably) including additional value added products and services with full-rate core products, while continuing to maintain profitability.

Pro forma financial statements for the first three years of operations are included. The following is a summary of key points:

[REDACTED]

\textbf{THE OFFERING}

Capital is needed to complete initial build out of the Sealand facility and ready it for commercial sales. This funding will be used to develop key technologies to make HavenCo's products and services even more attractive to target markets and to finance sales and marketing efforts.

As a first round of investment, HavenCo, Ltd. Is seeking up to [REDACTED] from angel investors within the infrastructure and client industries via an issue of Series A Preferred Shares. HavenCo, Ltd. will accept investment from individual investors in quantities of [REDACTED] or greater at a price of \$1.54 per share, placing the pre-money valuation of the company at [REDACTED]. This pre-money valuation includes a pool of [REDACTED] shares of common stock reserved for issue to future employees, consultants, and agents.

HavenCo may seek future rounds of financing in order to expand operations, further develop key technologies, and aggressively market its products and services. The extent of future rounds of financing is included in the pro forma financial statements, and is subject to change based on actual revenue levels.

HavenCo eventually plans to offer its shares publicly over the Internet, directly to investors, on its own stock exchange, allowing investors to profit financially, in a timely fashion.  Additionally, due to the superior tax situation afforded by Sealand incorporation, HavenCo may pay dividends without penalty.

 

 